\chapter{Related Work}
\label{chap:previous}
In rendering of subsurface scattering, most approaches rely on approximating correctly the \emph{Radiative Transfer Equation} (RTE). We identified two main approaches to the problem in literature:

\begin{description}
	\item[Analytical] One class of solutions consists of approximating the RTE or one of its approximations via an analytical model. These models can have different levels of complexity and computation times, and are often adaptable to a wide range of materials. However, often they rely on assumptions on the scattering parameters that limit their applicability.
	\item[Numerical] In this other class of solutions, a numerical approach is used instead of approximating the RTE with an analytical model. This methods include finite element methods and discrete ordinate methods, for which a numerical solution for the RTE is actually computed. While providing an exact solution, the computation times are longer. Other numerical approaches focus more on the appearance of the model and do not provide an exact solution for the RTE.
\end{description}

In this thesis, we focus on efficiently implementing a model that falls in the first category, the analytical models. In the following sections, we are going to describe approaches for each one of the mentioned categories, comparing them to our method.

\section{Analytical techniques}

In the analytical techniques, two different areas of research must be distinguished. The first area is the research on the actual models, while the second is research on how the actual models can be implemented efficiently. Each model is usually represented by a specific function called BSSRDF (\emph{Bidirectional Subsurface Scattering Reflectance Distribution Function}), that describes how light propagates between two points on the surface. An integration on the surface and all the directions must be performed in order to get how much light actually exits from a point on the surface. Implementation techniques focus on efficiently implementing this integration step, often making assumptions for which points the computation can be avoided. 

\subsection{Models}
Regarding the models, the first and most important is the dipole developed by \cite{Jensen:2001:PMS:383259.383319}. This models relies on an approximation of the RTE called the \emph{diffusion approximation}, which relies on the assumption of highly scattering materials. In this case, a BSSRDF for a planar surface in a semi-infinite medium can be obtained. The BSSRDF needs only the distance between two points to be calculated, and with some precautions can be also used with arbitrary geometry. This model does not include any single scattering term: it needs to be evaluated separately. The model has been further extended in order to account for thin object regions and multi-layered materials\citep{Donner:2005:LDM:1186822.1073308}.

A significant improvement of the model was later given by \cite{deondeon}, that improved the model to better fit path traced simulations without nearly any additional computation cost. A more advanced model based on quantization was proposed by \cite{D'Eon:2011:QMR:1964921.1964951}, that introduced a new physical foundation in order to improve the accuracy of the original diffusion approximation. Finally, some higher order approximation exist \citep{IMM2013-06646}, in order to account for the directionality of the incoming light and single scattering. This allows a more faithful representation of the model at the price of extended computation times. A comparison between the directional and the standard dipole can be seen in figure \ref{fig:comparison}.


\subsection{Implementations}

Most research on efficient implementations of a subsurface scattering analytical model has been made on the original model by \cite{Jensen:2001:PMS:383259.383319}. The first efficient implementation was proposed by \cite{Jensen:2002:RHR:566654.566619}, based on a two-pass hierarchical integration approach. Samples on the model are organized in an octree data structure, that then is used to render the object. In the first step, the radiance from the light is stored in the points. In the second pass, using the octree, the contribution from neighboring points is computed, clustering far points in order to speed up calculations. This approach can be adopted for the Jensen model, where the only parameter is the distance between the entering and exiting point. However, using the directional dipole, the samples cannot be clustered because of the directionality of the light: once we sum up the contribution from multiple lights, the contribution cannot be separated anymore. In fact, we would need a different clustering of the points for each light, that quickly becomes inefficient since whole octree would have to be stored on the GPU. 

\cite{Lensch:2002:IRT:826030.826632} approached the problem by subdividing the subsurface scattering contribution into two: a direct illumination part and a global illumination part (i.e. the light shining through the object). The global illumination part is pre-computed as vertex-to-vertex throughput and then summed to the direct illumination term in real-time. Compared to our method, this method requires a pre-computation step that depends on the geometry of the model, and thus deformation effects are not possible. Moreover, a coefficient has to be stored for each pair of vertex, that means a quadratic increase for increasing model size. Our method, on the other hand, occupies a memory space depending linearly on the number of vertices.

Translucent shadow maps \citep{Dachsbacher:2003:TSM:882404.882433} use an approach similar to standard shadow maps: they render the scene from the light point of view, and then calculate the dipole contribution in one point only from a selected set of points, according to a specified sampling pattern. As in \cite{Lensch:2002:IRT:826030.826632}, the contribution is split into global and local to permit faster computations. In our approach we will reuse some of the ideas introduced by translucent shadow maps: we will render the scene from the light point of view and we will reuse the information stored in the map such as depth, vertices and normals. However, our approach to using the values from the map is different from the original paper, as we will explain in chapter \ref{chap:method}. \cite{Mertens:2003:IRT:882404.882423} propose a fast technique based on radiosity hierarchical integration techniques, that unlike the previous implementation can handle deformable geometry.

Another important category of methods is screen space methods. \cite{1238246} propose an image space GPU technique that pre-computes a set of sample points for the area integration and then performs the integral over multiple GPU passes. \cite{d'Eon:2007:ERH:2383847.2383869,deonss} propose a method in image-space, interpreting subsurface scattering as a sum of images to which a gaussian filter has been applied. The gaussians are then summed with weights that make them fit the diffusion approximation. \cite{Jimenez:2009:SPR:1609967.1609970} improves further the technique, giving more precise results in case of skin. All these techniques assume that the diffusion profile can be pre-computed and then fitted with a sum of Gaussians: as we have already mentioned, this is not possible for the directional dipole, where the diffusion profile can change a lot depending on the angle of incidence of the incoming ray of light. Morevoer, even if we were able to compute the coefficients for each possible combination of parameters, it would not be possible to apply a gaussian filter with a different set of coefficients per point.

\cite{4736459} present a fast screen space technique that render the object as a series of splats, using GPU blending to sum over the various contributions. The diffusion profile in this case is pre-computed and stored as a texture. As in the previous techniques, the directionality of the incoming light does not allow the pre-computation of a diffusion profile. Moreover, the directional dipole is not symmetrical, so the splats would have to use a bigger radius in order to account for all the contribution, increasing computation costs.

\section{Numerical techniques}

Numerical techniques for subsurface scattering are often not specific, but come for free or as an extension of a global illumination numerical approximation, since the governing equations are essentially the same. Given their generality, they are usually slower than their analytical counterpart, and often rely on heavy pre-computation steps in order to achieve interactive framerates. The volumetric version of Jensen's Photon Mapping\citep{Jensen:1998:ESL:280814.280925} was originally developed to render participating media in general, but it has been adapted for subsurface scattering\citep{Dorsey:1999:MRW:311535.311560}. Classical approaches as a full Monte-Carlo simulation implementation of the light-material interaction, and finite-difference methods exist in literature\citep{raey}. 

Some less general methods have been introduced in order to devise more efficient approximations when it comes to the specific problem of subsurface scattering. \cite{raey} uses the diffusion approximation with the finite difference method on the object discretized on a 3D grid. \cite{Fattal:2009:PMI:1477926.1477933} uses as well a 3D grid, that is swept with a structure called light propagation map, that stores the intermediate results until the simulation is complete. All the numerical methods described so far are not real-time, and they are generally not feasible for a GPU environment. 

\cite{journals/cgf/WangWHSYG10}, instead of performing the simulation on a discretized 3D grid, makes the propagation directly in the mesh, converting it into a connected grid of tetrahedrons called \emph{QuadGraph}. This grid can be optimized to be GPU cache friendly, and provide a real-time rendering of deformable heterogeneous objects. The problem in this method is that the QuadGraph is slow to compute (20 minutes for very complex meshes) and has heavy memory requirements for the GPU. Compared to our method, this one requires an heavy pre-computation step, and allows only not-deformable objects. However, as most of propagation techniques, it can handle heterogeneous materials, while our method can not.

Precomputed radiance transfer methods are another class of general global illumination methods, that generally pre-compute part of the lighting and store it in tables\citep{Donner:2009:EBM:1531326.1531336}, allowing to retrieve it efficiently with an additional memory cost. The problem with this method compared to ours is that it requires a memory space that increases exponentially if we want to handle deformable materials and dynamic lighting. Moreover, it requires an heavy pre-computation in order to calculate the lighting coefficients. Our method, being analytical, does not required either a lot of memory or an hevay pre-computation step. 

A recent method called SSLPV - Subsurface Scattering Light Propagation Volumes \citep{Borlum:2011:SSL:2018323.2018325} extends a technique originally developed by \cite{Kaplanyan:2010:CLP:1730804.1730821} to propagate light efficiently in a scene using a set of discretized directions on a 3D grid. The method allows real-time execution times and deformable meshes with no added pre-computation step, with the drawback of not being physically accurate. Moreover, the required memory space on the GPU is larger than the one required than our method, since a voxelization of a mesh must be stored. 

Finally, for real-time critical applications (such as games), translucency is often estimated as a function of the thickness of the material, that is used to modify a lambertian term \citep{Tomaszewska2012,greenrtss}. A method by \cite{Kosaka:2012:RAR:2407156.2407206} uses an approach similar to the one we will describe in order to compute the thickness of the material using a different camera direction. While not physically accurate, this techniques allows to have a fast translucency effect that can be easily added to existing deferred pipelines. Compared to our method, this method requires no storage space and light computation. However, the risk is that the translucency effects are not represented faithfully, and some artifacts may appear, as pointed out in \cite{greenrtss}, and multisampling should be used in order to avoid artifacts. 

As we can see, in the reviewed literature so far there is not a proper way to account for direction in subsurface scattering in real-time. Given this literature review, we will introduce our method to handle directional subsurface scattering in real-time in chapter \ref{chap:method}. In the next chapter, we are going to give a theoretical introduction to a mathematical description of light transport, as well as giving the proper formulas and definition of the standard dipole model by \cite{Jensen:2001:PMS:383259.383319} and the directional dipole model presented by \cite{IMM2013-06646}.
