\chapter{Preface}

\vspace{1cm}
This thesis was prepared at the DTU Compute department at the Technical University of Denmark in fulfillment of the requirements for acquiring an M.Sc. in Digital Media Engineering. 

The thesis deals with the efficient rendering of translucent materials, using an innovative model proposed by the author's M.Sc. thesis supervisor, Jeppe Revall Frisvad. Translucent materials consist of a particular class of materials like fruit, marble, skin, and other materials where the effect of subsurface scattering cannot be neglected. 

The interest for real time rendering in the author arose during the course of his M.Sc. in Digital Media Engineering, where he focused on the study line in Computer Games. For this study line, he had to take several courses in real time computer graphics, and from this courses he got his interest in advanced shading techniques. In the spring 2014, professor Jeppe Revall Frisvad of DTU Compute proposed a research-oriented M.Sc. thesis, that had the final goal of creating a method for implementing the directional dipole in real time. The author deemed the topic to be a great opportunity to research in real time rendering techniques, and so he registered his application for this master thesis, \emph{\thesistitle}.

The thesis consists of a software implementation in C++, Qt and OpenGL, and this report. The initial Qt framework used was taken from DTU course 02564, \emph{Real Time Graphics}, and then expanded in order to fit the needs of the thesis. All the code reported in this document was written by the author an does not come from the original framework. All the screenshots in this document were generated using the developed software. Other images in the report, unless the original source is reported in the caption, were generated by the author.

%==================================================================================================
% SIGNATURE AREA
%==================================================================================================
\begin{table}[ht]
\begin{tabularx}{\textwidth}{X}
\vspace{1.5cm}
Lyngby, \thesishandin-\thesisyear \\
\hfill \thesisauthor \\
\hfill \includegraphics[scale=0.4]{images/sig.png} \\
\end{tabularx}
\end{table}